\chapter{Introduction}
In recent years, there have been a variety of research papers analyzing the different facets of open source software (OSS). OSS research is of interest because if OSS has the potential to challenge the economics and methods of commercial development, as some have claimed, it is important to understand and evaluate it\cite{mockus2002two}. The success of several prominent open source projects have shown that the OSS development model can be highly productive. For example, Godfrey and Tu \cite{godfrey2000evolution} found that the Linux kernel project experienced superlinear growth over its six-year lifespan, in spite of prior research asserting that growth of software systems in general tends to slow down as they become larger. Further studies have been performed to identify specific benefits associated with open source projects. Paulson, Succi, and Eberlein \cite{paulson2004empirical}, for instance, found empirical support for the hypothesis that "open source projects generally have fewer defects than closed source projects, as defects are found and fixed rapidly." The openness and decentralization of OSS projects can allow these projects to attain more developer attention than otherwise possible.

Although OSS projects are freely accessible to the developer community, it does not mean they are mutually exclusive with the interests of commercial organizations. In fact, a growing body of research has shown that the opposite is true. In a survey of developers of Java projects hosted on SourceForge, Capra, Francalanci, Merlo, and Lamastra discovered that 31\% of the projects had a firm that was contributing in some way, with an average of two firms contributing per project\cite{capra2009survey}. In some projects, elements of commercial and open source software development productively coexist; in a case study of Mozilla, an OSS project which incorporates some elements of traditional commercial software development processes, Mockus, Fielding, and Herbsleb \cite{mockus2002two} found that Mozilla had low defect density, comparable to that of the Apache project, which does not have such commercial influences.

Since it has become clear that commercial organizations are frequently participating in OSS projects, several studies have been undertaken to evaluate the effects of this involvement on the projects themselves. The motivation behind the present study is to develop a macroscopic view of how organizations interact with and contribute to OSS projects as a whole. Specifically, we had the following research questions:
\begin{itemize}
	\item Is commercial development of OSS highly centralized, in the sense that a few organizations are highly involved in OSS, while most others are on the periphery?
	\item Which organizations are most central to OSS development?
\end{itemize}
In order to answer these questions, data was collected from Apache Software Foundation (ASF) projects to measure contributions from individual organizations, which was then used to build a social network for these organizations. Chapter 2 describes the data collection process, and Chapter 3 describes the social network analysis. Chapter 4 describes the results of the study. Finally, Chapter 5 concludes.

\section{Research Topic and Experiment}
% TODO: stub
The time period \timeperiod{} under observation began at January 1, 2016 and ended at June 1, 2016. This period was chosen for two reasons:
\begin{itemize}
	\item A six-month period reduced the possibility of contributors having changed employers during the period, which would have introduced problems in cleanly associating their contributions with individual organizations.
	\item Since the LinkedIn employer data was collected in June 2016, having the period end at that time preserved the timeliness (and thus relevance) of that data.
\end{itemize}
