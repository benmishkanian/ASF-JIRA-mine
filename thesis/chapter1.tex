\chapter{Introduction}
In recent years, there have been a variety of research papers analyzing the different facets of open source software (OSS). OSS research is of interest because if OSS has the potential to challenge the economics and methods of commercial development, as some have claimed, it is important to understand and evaluate it\cite{mockus2002two}. The success of several prominent open source projects have shown that the OSS development model can be highly productive. For example, Godfrey and Tu \cite{godfrey2000evolution} found that the Linux kernel project experienced superlinear growth over its six-year lifespan, in spite of prior research asserting that growth of software systems in general tends to slow down as they become larger. Further studies have been performed to identify specific benefits associated with open source projects. Paulson, et al. \cite{paulson2004empirical}, for instance, found empirical support for the hypothesis that "open source projects generally have fewer defects than closed source projects, as defects are found and fixed rapidly." The openness and decentralization of OSS projects can allow these projects to attain more developer attention than otherwise possible.

Although OSS projects are freely accessible to the developer community, it does not mean they are mutually exclusive with the interests of commercial organizations. In fact, a growing body of research has shown that the opposite is true. In a survey of developers of Java projects hosted on SourceForge, Capra et al. discovered that 31\% of the projects had a firm that was contributing in some way, with an average of two firms contributing per project\cite{capra2009survey}.

The time period \timeperiod{} under observation began at January 1, 2016 and ended at June 1, 2016. This period was chosen for two reasons:
\begin{itemize}
	\item A six-month period reduced the possibility of contributors having changed employers during the period, which would have introduced problems in cleanly associating their contributions with individual organizations.
	\item Since the LinkedIn employer data was collected in June 2016, having the period end at that time preserved the timeliness (and thus relevance) of that data.
\end{itemize}
