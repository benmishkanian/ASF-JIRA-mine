\chapter{Organizational Social Networks}
\section{Application of Social Networks to OSS Research}
In research projects concerning OSS, social networks are a useful tool to build understanding of the overall structure and flow of the distributed collaboration. Madey, Freeh, and Tynan\cite{madey2002open}, for instance, used social networks to perform an empirical study of collaboration on SourceForge projects, finding that the number of developers per project and the number of projects per developer both have power-law distributions. Similarly, by modeling organizational collaborations in ASF projects as a social network, we can answer questions about the high-level scope and structure of organizational involvement in ASF.

For the present study, social networks were employed to find out the centralization level of organizational contributions to ASF, and to pinpoint the organizations which are most central to ASF project development. In addition to computing these values, the social networks were plotted using R, which provided an visual overview of the network structure. Afterwards, some of the most interesting phenomena observed in these network analyses were investigated in more detail, in order to better understand what caused them to appear, and what effects they have on the network.
\section{Building the Network}
Once the contributors' commit counts and employers were known, a social network of organizations was created based on the shared collaborations of their members. This was done in the following steps:
\begin{enumerate}
	\item The \textbf{companyprojectcommitcount} table was joined with itself on column \textbf{project} and the two commit counts of each row were summed into one, creating an edge list for companies, with the summed commit count as the edge weight.
	\item The edge list was imported into R, symmetric edges were removed, and edges connecting the same vertices were merged by summing their commit counts.
	\item Networks were created using R packages \textbf{network} and \textbf{igraph}.
\end{enumerate}
Essentially, there was an undirected edge between organizations A and B when a member of A and a member of B committed to the same project. The edge weight was the sum of the number of commits from A's members and the number of commits from B's members to all projects where they both contributed.

\section{Network Plots}
One challenge in plotting complex networks is that when there are many nodes and edges, it becomes difficult to discern the structure of a network by looking at the plot, especially if it was plotted using a simple algorithm. Since the organizational network had 90 vertices and 622 edges, a more sophisticated plotting approach was necessary. In their 2004 paper\cite{gonzalez2004community}, Gonz{\'a}lez-Barahona, L{\'o}pez, and Robles demonstrated that the Girvan-Newman (GN) algorithm can generate a hierarchical graph layout, which makes it easier to visually identify graph structure. They applied this method to plotting a network of developers of the Apache HTTP Server, the founding project of the ASF, and asserted that the GN algorithm provides ``a step forward at least in the aspect of offering a way to visually identify an affiliation network,'' compared to the classical network analysis algorithms used by previous works. Since the GN algorithm was shown to be effective for highlighting the structure of Apache developer network, it was applied for plotting the ASF organizational network.

An image of the plot is shown in figure \ref{fig:orgCommunities}. This plot was generated using the R package \verb|igraph|, which contains a function named \verb|edge.betweenness.community| that implements the GN algorithm to generate a ``communities'' object. Communities highlight graph structure by grouping vertices into different neighborhoods based on their edges. In the figure, each community is shaded with a different color, and vertices in the same community are colored the same, making it easier to see the relationships between clusters. Edge weights are not reflected in this graph, as the vertices could not be spread out far enough to allow increasing edge thickness without causing disruption.

In looking at the graph, it is immediately apparent that there are identifiable clusters of organizations that are in close collaboration, and there are a few ``linchpin'' organizations which connect these clusters. Although the center cluster is still a bit too cluttered, the easy identifiability of the communities and the vertices connecting them provides further evidence that hierarchical algorithms such as GN are well-suited for the task of exhibiting the structure of complex social networks. A closer analysis of the clusters is provided in section \ref{clustersection}.
\begin{figure}
	\includegraphics[width=\textwidth]{communities.png}
	\centering
	\caption{The organizational communities}
	\label{fig:orgCommunities}
\end{figure}

\section{Centrality and Centralization Measures}
Measuring the centrality of nodes in a network can be an effective way to learn about the network structure. As a review, the following discussion includes definitions of centrality and centralization, taken from \textit{Introduction to social network methods}\cite{hanneman}.

The centrality of a vertex is a measure of how close it is to the ``center'' of action in a network. There are multiple different ways to measure centrality, but the most common are degree centrality, closeness centrality, and betweenness centrality. Each of these was computed for all vertices in the organizational networks. The next three subsections provide more detail on these measures, followed by a discussion on the centralization of the network as a whole.

\subsection{Degree Centrality}
Degree centrality is simply measuring centrality using the degree (i.e. the number of incident edges) of a vertex. For directed graphs, degree can be measured using indegree, outdegree, or the sum of the two. Since the organizational network is an undirected graph, all edges incident to a vertex were counted in the degree. A table of the organizations with the highest degree centrality is shown in table \ref{tab:degree}.

\begin{table}
	\begin{tabular}{l|c}%
		\bfseries Organization & \bfseries Degree% specify table head
		\csvreader[head to column names]{degree.csv}{}% use head of csv as column names
		{\\\hline\organizationa & \scorea}% specify your coloumns here
	\end{tabular}
	\centering
	\caption{The 10 organizations with the highest degree centrality}\label{tab:degree}
\end{table}
\subsection{Closeness Centrality}
Closeness centrality is a measure of how ``close'' a vertex is to all others in the network. For the present study, the closeness of two vertices was computed using the following formula:
\begin{equation*}
	C(v) = (|V(G)|-1)/sum( d(v,i), i in V(G), i!=v )
\end{equation*}
where where d(i,j) is the geodesic distance between i and j (where defined)\cite{butts}.

A table of the organizations with the highest closeness centrality is shown in table \ref{tab:closeness}. Unlike degree centrality, closeness centrality takes into account the distances to vertices that are more than one edge away. This means that vertices raking high in closeness centrality are central to the network as a whole, rather than simply central to their neighborhoods. For example, SK Telecom has low closeness, but high degree. Looking back at the visualization in figure \ref{fig:orgCommunities}, we can see this is because it is directly in the center of the large central cluster, which provides it many direct edges, but insulates it from all other clusters. In the other extreme, Fitbit has high closeness but relatively low degree, because although it has short paths to most clusters, it achieved this through a small number of links to other linchpin organizations. Still, eight out of the ten organizations are included in both lists, so already a pattern begins to emerge about which organizations are most central overall.

\begin{table}
	\begin{tabular}{l|c}%
		\bfseries Organization & \bfseries Closeness% specify table head
		\csvreader[head to column names]{closeness.csv}{}% use head of csv as column names
		{\\\hline\organizationb & \scoreb}% specify your coloumns here
	\end{tabular}
	\centering
	\caption{The 10 organizations with the highest closeness centrality}\label{tab:closeness}
\end{table}

\subsection{Betweenness Centrality}
Betweenness centrality is a measure of how frequently a vertex appears in the list of shortest paths between all other vertices. Stated differently, it measures how likely it is that vertex V must be traversed when taking the shortest path between arbitrary vertices A and B. Betweenness places more emphasis on the importance of edges than closeness does; an edge that is on many shortest paths is likely to confer high between centrality scores on its incident vertices.

A table of the organizations with the highest betweenness centrality is shown in table \ref{tab:betweenness}. One change that immediately stands out is that Red Hat is ranked second, whereas in the other two rankings it was not even in the top ten. This can be explained by taking another look at figure \ref{fig:orgCommunities}. At first glance, Red Hat seems to have similar positioning to Fibit, in that it has a medium number of edges to some other important actors. However, the difference is that Fitbit's direct connections are all in the 3 clusters nearest to it, whereas Red Hat has far-reaching connections to several clusters on the other side of the graph. This means that the shortest paths between nodes at the periphery of the graph are much more likely to require use of Red Hat's edges, because it is an efficient bridge between clusters, compared to Fitbit's edges.

% This does not escape special characters--be careful with ``Ampool, Inc.''
\begin{table}
	\begin{tabular}{l|c}%
		\bfseries Organization & \bfseries Betweenness% specify table head
		\csvreader[head to column names]{betweenness.csv}{}% use head of csv as column names
		{\\\hline\organizationc & \scorec}% specify your coloumns here
	\end{tabular}
	\centering
	\caption{The 10 organizations with the highest betweenness centrality}\label{tab:betweenness}
\end{table}

\subsection{Centralization}
Centralization measures the amount of variability in the centrality of the vertices of a network. It is a statistic of the network, rather than the individual nodes. A network with high centralization is one in which there is an unequal distribution of centrality in different parts of the network.  Centrality was computed for the organizational network using the following formula, developed by Freeman\cite{freeman1978centrality}:
\begin{equation*}
C^*(G) = sum( |max(C(v))-C(i)|, i in V(G) )
\end{equation*}
This formula requires some function C to measure centrality of vertices, and since there were three available methods to measure centrality, there are three different computed values for the network centralization. Table \ref{tab:centralization} displays the centralization value for each centrality measure. These values can be interpreted as percentages of centralization, with 1.00 indicating a completely centralized graph where only one vertex is central.

The difference between the centralization values computed when using closeness versus betweenness for measuring centrality is striking. This discrepancy was likely caused by the fact that most of the vertices had a betweenness value of zero, making most of the summands in the centralization formula zero as well. Indeed, from the perspective of betweenness centrality, the network truly does appear somewhat decentralized, because the majority of the nodes are not on any shortest paths, and thus they are equally low in betweenness centrality. However, the nodes that are high in betweenness centrality have extremely high scores it it, keeping the network centralization above 25\%. Due to the skewedness of the betweenness centrality distribution, this is likely somewhat of an underestimate. After all, a visual inspection of the graph makes it readily apparent that the central cluster would result in near complete centralization, if nodes such as Red Hat and Fitbit did not provide alternative bridges between clusters. From the three measures, we can compute an average centralization value of 43\%, and we can conclude that the centralization level of the organizational network is somewhere between 25\% and 50\%, which is moderately centralized.

\begin{table}
	\begin{tabular}{l|c}
		\bfseries Centrality Type & \bfseries Centralization \\
		\hline
		Degree & 0.4905946 \\
		\hline
		Closeness & 0.5550736 \\
		\hline
		Betweenness & 0.2619162
	\end{tabular}
	\centering
	\caption{The network centralization value for each centrality measure}\label{tab:centralization}
\end{table}
\section{Cluster Analysis}\label{clustersection}