\chapter{Conclusion}
This thesis presented several findings about how organizations contribute to Apache Software Foundation projects. A social network was used to model the collaborations between organizations, allowing both visual and mathematical analysis of the structure of the ASF from an organizational perspective. The network was found to be moderately centralized, with a central cluster of Hadoop related collaboration being one of the main features of the network. It was also shown that a few organizations achieved high centrality not by existing in the central cluster, but by connecting multiple disparate clusters.

%\pd{Some discussion of why this matters. E.g., useful to know what the clusters of related businesses are; which businesses play central roles overall, and in each cluster; which businesses are playing "gateway" or brokering roles. This important could be useful for both people considering use or actually using this software (why?) for investors (why) for job-seekers (why?) and for employers (why?)}. \bm{Added.}

\pd{Actually add comments relating to the importance of your conclusions, as related above,  at the appropriate places in chapter 4}. 

These findings have implications for individuals and organizations that are interested in the ASF. First of all, it was shown that organizations play a very significant role in the ASF, with some projects having substantially all of their commits during the observed period authored by employees of a single organization. Second, the centralization of the organizational network was discovered to be moderate, meaning that there may be a significant disparity in knowledge flow or other benefits experienced by two organizations contributing in different areas of the ASF. Third, it was revealed that the most central cluster of organizations is closely related to development of the Hadoop ecosystem, which provides insight into how much Hadoop has become central to the ASF as a whole; work on distributed computing in general comprised a large chunk of ASF development in 2016 thus far.

In addition to the above, the organizational social networks have practical use for several different groups of people. Potential users of ASF software can use these networks to find out which organizations are supporting the software, to get an idea of the amount and quality of support to expect. Investors can use them to identify high-growth domains of software, helping them get ahead of trends. Job-seekers can use them to find out which skills are in-demand, perhaps even specifically at the company they want to work at. Employers can use them to identify qualified candidates by observing contributors to related projects. These are just a few examples; in general, there is value in understanding the structure of social networks when one is (or may become) a stakeholder in that network, and the organizational social network of ASF development is no exception.

There are several avenues through which this work could be improved or extended. To begin with, a primary limitation of this study was that it only considered one epoch of time in which to analyze social networks. The next logical step would be to do a longitudinal study to find out how the data and networks change as time passes. In addition, there were several phenomena revealed which merit further experimentation or case studies. For example, it would be useful to find out why there are so many projects with only one organization contributing substantially all of the commits, and how the outcomes for these projects compare to that of the others. Another missing piece of information is the extent to which shared collaboration within a project results in knowledge flow or other social benefits. Without some concrete evidence on this, we can not precisely characterize the magnitude of the importance of centrality in this network. Finally, as with many other studies of individual OSS communities, it remains to be proven that the results found for ASF social networks are applicable to OSS in general.