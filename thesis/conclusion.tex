\chapter{Conclusion}
This thesis presented several findings about how organizations contribute to Apache Software Foundation projects. A social network was used to model the collaborations between organizations, allowing both visual and mathematical analysis of the structure of the ASF from an organizational perspective. The network was found to be moderately centralized, with a central cluster of Hadoop related collaboration being one of the main features of the network. It was also shown that a few organizations achieved high centrality not by existing in the central cluster, but by connecting multiple disparate clusters.

\pd{Some discussion of why this matters. E.g., useful to know what the clusters of related businesses are; which businesses play central roles overall, and in each cluster; which businesses are playing "gateway" or brokering roles. This important could be useful for both people considering use or actually using this software (why?) for investors (why) for job-seekers (why?) and for employers (why?)}. 

\pd{Actually add comments relating to the importance of your conclusions, as related above,  at the appropriate places in chapter 4}. 

These findings have implications for how individuals and organizations that are interested in the ASF. First of all, it was shown that organizations play a very significant role in the ASF, with some projects having substantially all of their commits during the observed period authored by employees of a single organization. Second, the centralization of the organizational network was discovered to be moderate, meaning that there may be a significant disparity in knowledge flow or other benefits experienced by two organizations contributing in different areas of the ASF. Third, it was revealed that the most central cluster of organizations is closely related to development of the Hadoop ecosystem, which provides insight into how much Hadoop has become central to the ASF as a whole; work on distributed computing in general comprised a large chunk of ASF development in 2016 thus far.

% TODO: Future work:
% - Take into account the time component (split into epochs, see changes over time).