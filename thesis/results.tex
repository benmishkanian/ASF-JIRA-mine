\chapter{Results}
This chapter reviews the results of the experiment, also providing interpretations which could be used as hypotheses for further studies. The answers to the research questions posed in chapter 1 are also discussed here.

\pd{Ben, I think it's better if we move centralization after centrality. that section is stronger} \bm{Agreed. I swapped them.}

\section{Central Organizations}
One primary goal of this study was to find out which organizations are central to OSS development. Since ASF is only one community of the broader OSS development community, in order to apply the results of the present experiment to OSS in general, an assumption must be made that ASF has similar organizational dynamics to most other OSS communities. As this has not yet been proven, the following results are presented as pertaining to ASF in particular. Three different methods of calculating centrality were applied to the nodes in the organizational network, and the full results are shown in the tables in appendix \ref{ch:centralities}. By each measure, Hortonworks is the most central organization, but there was some variation for the rest of the results.

\pd{Use degree centrality and between centraily as subsction headings to discussion the two} \pd{Added the new sections.}
\subsection{Degree/Closeness Centrality}
Similar rankings were obtained when using degree centrality and closeness centrality. The reason for this is the structure of the central cluster. Nodes in this cluster have high closeness centrality because the cluster itself is central---there is a relatively short path from this cluster to any other cluster, and by extension, any other node. In addition, nodes in this cluster have high degree centrality because the organizations tend to contribute to many of the projects associated with the cluster, resulting in a dense subgraph in this area. Effectively, the density within the cluster provides short paths to the periphery of the cluster, and the centrality of the cluster itself provides short paths from the periphery of this cluster to other ones, resulting in high scores in both degree and closeness centrality for the cluster's members. Indeed, all of the top ten nodes having highest degree are in this cluster, as are nine of the top ten having highest closeness (with Fitbit being the outlier, although it remains in tenth place.)

\subsection{Betweenness Centrality}
The results for betweenness centrality were significantly different from those of the other two. This is because the betweenness formula rewards nodes which are incident to edges that are on many geodesics between arbitrary nodes, whereas the closeness formula rewards nodes that have short geodesics for themselves. In other words, even if a node has relatively short paths to most others, it may still have low betweenness, if these paths are only convenient for the node itself. A node with high betweenness typically has uniquely valuable edges that serve to connect regions that would otherwise have longer geodesics if these edges did not exist. Thus, from a social network perspective, the actor with high betweenness is one whose connections are considered valuable to many \textit{other} actors. In the organizational network, this was primarily Hortonworks and Red Hat.

The centrality of Hortonworks is investigated further in the next section, but its high betweenness is linked to its high degree and the centrality of its cluster. Although it does not seem to have any one edge that looks particularly valuable, it does have a large number of edges linking it to the nearby clusters, due to the fact that it contributed to 17 projects, the most of any organization in the dataset. This can make it an efficient gateway from within the tightly connected central cluster to the other clusters. That being said, what is more interesting is Red Hat's high betweenness, particularly since it lacks the number of connections and central cluster positioning that Hortonworks enjoys. It achieves this by having a few important edges that connect disparate clusters directly. For example, it has the only edge that directly connects its own cluster with Ampool's cluster. Even more significantly, it has the only edges that prevent the Citrix cluster from becoming a disconnected subgraph. Further analysis on why Red Hat has such valuable edges is provided in the next section.

The high betweenness centralities of IBM and Ampool have similar explanations to those of Hortonworks and Red Hat, respectively. IBM is third in project count, and it occupies a similar network location to Hortonworks, so it too achieves high betweenness by having many edges to nearby clusters. In the case of Ampool, just as Red Hat is the only bridge to the Citrix cluster, Ampool is the only bridge to its own cluster. In fact, by inspecting the network plot, it is evident that most of the top ten organizations in betweenness cleanly fall into one of these two archetypes: they are either bridges within clusters, or bridges between clusters.

\subsection{Case Studies of Central Organizations}
As a case study, Hortonworks' contribution patterns were examined in more detail to find out why it is so central. One of Hortonworks' core products, Hortonworks Data Platform, is built on top of 22 ASF projects\cite{hdp}, which explains why it has so many connections in the network. In addition, these projects are closely related in terms of domain and usage. Table \ref{tab:centralclusterprojects} summarizes the primary purpose of each of the projects associated with the central cluster, of which Hortonworks is a member. Clearly, they are all ``Hadoop related'' projects\cite{hadoopecosystem}, supporting the Hadoop ecosystem of large-scale distributed data processing.

\begin{table}
	\begin{tabular}{l|l}
		\bfseries Project & \bfseries Purpose \\
		\hline
		spark & large-scale data processing engine\cite{spark} \\
		storm & distributed realtime data processing\cite{storm} \\
		kafka & scalable distributed messaging system\cite{kafka} \\
		oozie & ``workflow scheduler system to manage Apache Hadoop jobs''\cite{oozie} \\
		bigtop & ``packaging, testing, and configuration\textellipsis{}of big data components''\cite{bigtop} \\
		flink & ``distributed stream and batch data processing''\cite{flink} \\
		hbase & ``distributed, scalable, big data store''\cite{hbase} \\
		samza & ``distributed stream processing framework''\cite{samza}
	\end{tabular}
	\centering
	\caption{The projects associated with the central cluster}\label{tab:centralclusterprojects}
\end{table}

\pd{``timeperiod" should be the actual dates rather than sym bolic}\bm{Fixed.}
Hortonworks was the only organization that contributed to all of these projects between January and June 2016, but several of the organizations with the highest centralities contributed to many of them. For example, Confluent and IBM each contributed to five of the eight projects, and Cloudera and Yahoo each contributed to four. Additionally, most of the organizations that provide the largest numbers of active contributors are primarily contributing to these projects. Evidently, a significant chunk of the total contemporary development effort in ASF is focused on the Hadoop ecosystem, both in terms of the number of organizations contributing, and also the number of contributors per organization. Due to the overall centrality of the Hadoop related organizational cluster, the type of projects an organization contributes to is currently a big factor in determining its centrality. However, since this experiment observed only one section of time, it is unknown whether this type of domain-dominated centrality was prevalent in the past, or will continue to be in the future.

As stated previously, the existence of ``bridge'' organizations such as Red Hat helps to lower the network centralization, because they provide short paths between clusters. The question is, why do these organizations contribute to projects not associated with their own clusters, while most of the others generally stick to projects associated with their clusters? Digging deeper into Red Hat's contributions helped to shed some light on the answer: Red Hat sells private cloud infrastructure-as-a-service\cite{redhat}, so unsurprisingly it contributed to cloud related ASF projects such as cloudstack and libcloud. Cloud computing is a different domain from the distributed computing promoted by the Hadoop ecosystem, but it shares some of the scalability challenges. This may be why Red Hat also contributed to flink and mahout, two projects which support scalable computing. In doing so, it became a bridge between the cloud related cluster and the Hadoop related one. Further research is necessary to evaluate whether projects with broader domains truly are catalysts for increasing knowledge flow between organizational clusters. \pd{Add some discussion of the other top organization, like ampool and ibm. Ampool is interesting becuase it's not a topper in other centrality measures} \bm{I added the explanation to the betweenness section.}

\pd{some discussion of closeness centrality. }\bm{I included some of it in the above sections.}

\section{Centralization}
Another primary research questions of this study was whether commercial development of OSS is highly centralized. The centralization of the organizational network was estimated to be between 0.25 and 0.50, based on three different measures of centrality. In comparison, Crowston and Howison\cite{Crowston2006} studied interaction networks of developers in bug tracking systems on SourceForge, ASF, and Savannah, finding a wide range of centralizations within different projects, ranging from near 0 to near 1, with a mean of 0.54. In light of their findings that extreme values of centralization and decentralization are commonplace in OSS social networks, the centralization values of the organizational network can be considered moderate in this context.

One way to interpret high centralization in a network is to say the ``balance of power'' in the network is not evenly distributed. There are potential advantages to being a central node in a centralized network. For example, if the organizational network's edges are considered to represent the flow of knowledge between organizations, then all knowledge flowing between JDriven and the other clusters flows through Ampool first. This gives Ampool power over JDriven, because it has access to knowledge external to their cluster, and it may or may not share that knowledge. Stated in more concrete terms, Ampool has acquired knowledge by contributing to project bigtop, and it also contributes to project groovy, along with JDriven; however, Ampool is the only contributor to groovy that also contributes to other projects, so Ampool has the potential to be a valuable source of knowledge to the groovy contributors, due to its broader knowledge of other projects. If Ampool shares new information about tools, design patterns, or methodologies, for instance, it can aid groovy development. Although such information is not wholly inaccessible to the other groovy contributors otherwise, Ampool's experience with other projects can facilitate the knowledge flow. Further experiments are necessary to evaluate the extent to which such knowledge flow takes place, and what other variable affect it. Even so, it can be assumed that centrality in the network is advantageous to organizations because it at least provides them the \textit{potential} of increased access to information, regardless of whether or not it is realized.

Since the organizational network is moderately centralized, it implies there exist certain disadvantages for organizations that exist on the periphery of the network. These organizations are typically on the outskirts because they contribute to only one or two projects, where there are few other organizations contributing. Such isolation from other organizations can make the aforementioned knowledge sharing difficult. In turn, this can reduce the visibility of the projects, an instance of the ``rich get richer'' effect that is often associated with centralized networks. This is not to say that it is always better to contribute to the ``popular'' projects; rather, it is a suggestion that there is value in contributing to multiple projects, even if the additional projects are only indirectly related to the organization's operations. The fact that all clusters in the network are connected shows that some organizations have found reasons to contribute to projects that the other members of their cluster ignored. In doing so, they became linchpins. Of course, some organizations may not have the resources to contribute to multiple projects, but for those that do, the potential benefits of doing so in this centralized network should be taken into consideration.

\pd{Add subsection with Discussion about clusters} 