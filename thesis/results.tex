\chapter{Results}
This chapter reviews the results of the experiment, also providing interpretations which could be used as hypotheses for further studies. The answers to the research questions posed in chapter 1 are also discussed here.

\section{Centrality and Centralization}
One of the primary research questions of this study was whether commercial development of OSS is highly centralized. Since ASF is only one community of the broader OSS development community, in order to apply the results of the present experiment to OSS in general, an assumption must be made that ASF has similar organizational dynamics to most other OSS communities. As this has not yet been proven, the following results are presented as pertaining to ASF in particular.

The centralization of the organizational network was estimated to be between 0.25 and 0.50, based on three different measures of centrality. In comparison, Crowston and Howison\cite{Crowston2006} studied interaction networks of developers in bug tracking systems on SourceForge, ASF, and Savannah, finding a wide range of centralizations within different projects, ranging from near 0 to near 1, with a mean of 0.54. In light of their findings that extreme values of centralization and decentralization are commonplace in OSS social networks, the centralization values of the organizational network can be considered moderate in this context.