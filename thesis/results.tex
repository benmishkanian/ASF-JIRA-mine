\chapter{Results}
This chapter reviews the results of the experiment, also providing interpretations which could be used as hypotheses for further studies. The answers to the research questions posed in chapter 1 are also discussed here.

\section{Centralization}
One of the primary research questions of this study was whether commercial development of OSS is highly centralized. Since ASF is only one community of the broader OSS development community, in order to apply the results of the present experiment to OSS in general, an assumption must be made that ASF has similar organizational dynamics to most other OSS communities. As this has not yet been proven, the following results are presented as pertaining to ASF in particular.

The centralization of the organizational network was estimated to be between 0.25 and 0.50, based on three different measures of centrality. In comparison, Crowston and Howison\cite{Crowston2006} studied interaction networks of developers in bug tracking systems on SourceForge, ASF, and Savannah, finding a wide range of centralizations within different projects, ranging from near 0 to near 1, with a mean of 0.54. In light of their findings that extreme values of centralization and decentralization are commonplace in OSS social networks, the centralization values of the organizational network can be considered moderate in this context.

One way to interpret high centralization in a network is to say the ``balance of power'' in the network is not evenly distributed. There are potential advantages to being a central node in a centralized network. For example, if the organizational network's edges are considered to represent the flow of knowledge between organizations, then all knowledge flowing between JDriven and the other clusters flows through Ampool first. This gives Ampool power over JDriven, because it has access to knowledge external to their cluster, and it may or may not share that knowledge. Stated in more concrete terms, Ampool has acquired knowledge by contributing to project bigtop, and it also contributes to project groovy, along with JDriven; however, Ampool is the only contributor to groovy that also contributes to other projects, so Ampool has the potential to be a valuable source of knowledge to the groovy contributors, due to its broader knowledge of other projects. If Ampool shares new information about tools, design patterns, or methodologies, for instance, it can aid groovy development. Although such information is not wholly inaccessible to the other groovy contributors otherwise, Ampool's experience with other projects can facilitate the knowledge flow. Further experiments are necessary to evaluate the extent to which such knowledge flow takes place, and what other variable affect it. Even so, it can be assumed that centrality in the network is advantageous to organizations because it at least provides them the \textit{potential} of increased access to information, regardless of whether or not it is realized.

Since the organizational network is moderately centralized, it implies there exist certain disadvantages for organizations that exist on the periphery of the network. These organizations are typically on the outskirts because they contribute to only one or two projects, where there are few other organizations contributing. Such isolation from other organizations can make the aforementioned knowledge sharing difficult. In turn, this can reduce the visibility of the projects, an instance of the ``rich get richer'' effect that is often associated with centralized networks. This is not to say that it is always better to contribute to the ``popular'' projects; rather, it is a suggestion that there is value in contributing to multiple projects, even if the additional projects are only indirectly related to the organization's operations. The fact that all clusters in the network are connected shows that some organizations have found reasons to contribute to projects that the other members of their cluster ignored. In doing so, they became linchpins. Of course, some organizations may not have the resources to contribute to multiple projects, but for those that do, the potential benefits of doing so in this centralized network should be taken into consideration.

\section{Central Organizations}
Another goal of this study was to find out which organizations are central to OSS development. Three different methods of calculating centrality were applied to the nodes in the organizational network, and the full results are shown in the tables in appendix \ref{ch:centralities}. By each measure, Hortonworks is the most central organization, but there is some variation for the rest of the results.

As a case study, Hortonworks' contribution patterns were examined in more detail to find out why it is so central. One of Hortonworks' core products, Hortonworks Data Platform, is built on top of 22 ASF projects\cite{hdp}, which explains why it has so many connections in the network. In addition, these projects are closely related in terms of domain and usage. Table \ref{tab:centralclusterprojects} summarizes the primary purpose of each of the projects associated with the central cluster, of which Hortonworks is a member. Clearly, they are all ``Hadoop related'' projects\cite{hadoopecosystem}, supporting the Hadoop ecosystem of large-scale distributed data processing.

\begin{table}
	\begin{tabular}{l|l}
		\bfseries Project & \bfseries Purpose \\
		\hline
		spark & large-scale data processing engine\cite{spark} \\
		sqoop & data transfer between Hadoop and databases\cite{sqoop} \\
		storm & distributed realtime data processing\cite{storm} \\
		kafka & scalable distributed messaging system\cite{kafka} \\
		knox & REST API gateway for Hadoop\cite{knox}
	\end{tabular}
	\centering
	\caption{The projects associated with the central cluster}\label{tab:centralclusterprojects}
\end{table}

% TODO: find out if the most central orgs contributed to all 5 central cluster projects